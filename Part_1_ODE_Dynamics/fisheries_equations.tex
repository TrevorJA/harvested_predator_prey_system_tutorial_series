\documentclass{article}
\usepackage[utf8]{inputenc}
\usepackage{hyperref}
\hypersetup{
    colorlinks=true,
    linkcolor=blue,
    filecolor=magenta,      
    urlcolor=cyan,
    pdftitle={Overleaf Example},
    pdfpagemode=FullScreen,
    }

\title{Fisheries Tutorial Equations}
\author{Trevor}
\date{Summer 2022}

\begin{document}
\maketitle

Derivation of stability conditions. Taken from \href{https://waterprogramming.wordpress.com/2017/09/22/exploring-the-stability-of-systems-of-ordinary-differential-equations-an-example-using-the-lotka-volterra-system-of-equations/}{Antonia's 2017 blog post}.


Generalized forms of the ODEs. 
\bigbreak For the prey population:
$$\frac{dx}{dt} = F(x)$$

For the predator population:
$$\frac{dy}{dt} = G(x)$$

\section{Original Lotka-Volterra Form}
The original Lotka-Volterra equations:
\bigbreak For the prey population:
$$\frac{dx}{dt} = xb - axy$$

For the predator populations
$$\frac{dy}{dt} = caxy - dy = y(cax - d)$$

\bigbreak For a, b, c, d $>$ 0 are the parameters describing the
growth, death, and predation of the fish.

\section{Modified Lotka-Volterra:}
We use the modified system of equations proposed by Arditi and Akcakaya, which is a predator-dependent (dependent upon both predator and prey populations) equation which accounts for interactions between predators:
\bigbreak For the prey population:
$$\frac{dx}{dt} = bx(1-\frac{x}{K}) - \frac{\alpha xy}{y^m + \alpha hx}$$

\bigbreak The predator population is:
$$ \frac{dy}{dt} = \frac{c\alpha xy}{y^m + \alpha hx} -dy $$

\section{Stability}
The equilibrium condition is defined as the point in which the predator and prey populations do not change, i.e.,:
$$\frac{dx}{dt} = \frac{dy}{dt} = 0$$

\bigbreak Here, we are interested in the non-trivial stability conditions (i.e., $x \neq 0$ and $y \neq 0$). 

\subsection{Predator Stability}
Solving for the predator stability, $\frac{dy}{dt} = 0$:
$$\frac{c\alpha xy}{y^m + \alpha hx} -dy = 0$$

Re-arrangement of equation yields:
$$c\alpha xy = d(y^m + \alpha hx)$$

$$\alpha x(c-dh) - dy^m = 0$$

Solving for $x$ yields the predator zero-isocline: 

$$x^* = \frac{dy^m}{\alpha (c - dh)}$$

The stability condition is then defined as the condition under which this isocline is non-zero. In order for this to be non-zero, the condition is:

$$c > hd $$

\subsection{Prey Stability}
Now, solving for the prey isocline, where $\frac{dx}{dt} = 0$:



\end{document}
